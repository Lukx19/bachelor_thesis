\chapter*{Introduction}
\addcontentsline{toc}{chapter}{Introduction}
Humanity has envisioned many tasks which could be carried out by robots including transportation, health care, save and rescue and many more. Robots of current world can efficiently operate only in very limited conditions. To solve problems of the future we need fast and reliable algorithms for our robots. Big question in field of mobile robotics is how to efficiently localize robot and create map as precise as possible. This problem is often referred to as \gls{SLAM} problem.

Good localization is crucial part of any good navigation software. Generated map plays important role in path planning and  multi-robot coordination. SLAM algorithm should relay mostly on robots internal sensors like, e.g., sonars, cameras, wheel encoders. Using \gls{GPS} is only possible in outdoor environment. Precision of this localization is very often not good enough to successfully navigate robot.

The solution to full problem of map building and robot positioning needs to combining algorithms for map representation, sensor measurement registration and position estimation. This work presents novel approach in full \gls{SLAM} problem based on \gls{NDT} maps . In recent years \gls{NDT} map building process has proven to be reliable choise for scan registration. A map representation based on \gls{NDT} can handle dynamic objects and efficient occupancy update. The pose estimation problem was in recent years solved mostly by graph-based \gls{SLAM} optimizing engines. The graph based method offers flexibility and speed even on big maps. Both techniques were studied separately and providing great results. The missing part is how to combine these approaches to improve robustness of full \gls{SLAM} solution. In order to fulfill this goal we will present novel method  for roust registration on top of \gls{NDT} grids. The most challenging part of this fusion is how to represent the map. We use method based on small local mini maps which are easily used in graph of the \gls{SLAM} optimizer. Our algorithm has additional robustness to odometry error by utilizing our \gls{NDT} version of incremental scan matching. Combination of these part creates whole system which can estimate its position without initial guess and robustly close errors caused by imprecise robot movement. On top of algorithm benefits, we wanted to make source code and implementation easily accessible and improvable. For this reason we have decided to implement it in \gls{ROS}, which is current standard environment for robotic projects of all sizes.

This work has following structure. First chapter analyze full \gls{SLAM} on \gls{NDT} maps. Second chapter provides more information about algorithms used in this work. Third chapter describes whole system of NDT SLAM. Fourth chapter makes focus on implementation details behind the algorithms. In the last chapter we wrap up results of this algorithm and compare it to existing \gls{ROS} implementations.  

