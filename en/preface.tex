\chapter*{Introduction}
\addcontentsline{toc}{chapter}{Introduction}
Humanity has envisioned many tasks which could be carried out by robots including transportation, health care, save and rescue and many more. Robots of current world can efficiently operate only in very limited conditions. To solve problems of the future we need fast and reliable algorithms for our robots. Big question in field of mobile robotics is how to efficiently localize robot and create map as precise as possible. This problem is often referred to as \gls{SLAM} problem.

Good localization is crucial part of any good navigation software. Generated map plays important role in path planning and  multi-robot coordination. SLAM algorithm should relay mostly on robots internal sensors like, e.g., sonars, cameras, wheel encoders. Using \gls{GPS} is only possible in outdoor environment. Precision of this localization is very often not good enough to successfully navigate robot.

In last decade many high quality \gls{SLAM} algorithms where presented. Solution to full problem of map building and robot positioning is usually done by combining algorithms for map representation, sensor measurement registration and position estimation. This work solves full \gls{SLAM} problem by combination of \gls{NDT} map building process and scan registration with well developed research branch of graph-based \gls{SLAM} optimizing engines. To achieve this combination this work presents extended implementation of \gls{NDT} mapping process and new way of robust scan-matching on \gls{NDT} map. Implementation is done in \gls{ROS} to make it easy to test, use and improve.

This work has following structure. First chapter analyze full \gls{SLAM} on \gls{NDT} maps. Second chapter provides more information about algorithms used in this work. Third chapter describes whole system of NDT SLAM. Fourth chapter makes focus on implementation details behind the algorithms. In the last chapter we wrap up results of this algorithm and compare it to existing \gls{ROS} implementations.  

