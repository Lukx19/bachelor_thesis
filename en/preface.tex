\chapter*{Introduction}
\addcontentsline{toc}{chapter}{Introduction}
Humanity has envisioned many tasks which could be carried out by robots including transportation, health care, save and rescue and much more. Robots of the current world can efficiently operate only in very limited conditions. To solve problems of the future we need fast and reliable algorithms for our robots. The big question in the field of mobile robotics is how efficiently localize a robot and create the map as precise as possible. This problem is often referred to as \gls{SLAM} problem.

Precise localization is a crucial part of any good navigation software. Generated map plays an important role in path planning and multi-robot coordination. SLAM algorithm should rely mostly on robots internal sensors like, e.g., sonars, cameras, wheel encoders. Using \gls{GPS} is only possible in an outdoor environment. The precision of this localization is very often not good enough to successfully navigate robot.

The solution to the full problem of map building and robot positioning needs to combine algorithms for map representation, sensor measurement registration and position estimation. This work presents a novel approach in full \gls{SLAM} problem based on \gls{NDT} maps. In recent years \gls{NDT} map building process has proven to be a reliable choice for scan registration. A map representation based on \gls{NDT} can handle dynamic objects and updates occupancy. The pose estimation problem was in recent years solved mostly by graph-based \gls{SLAM} optimizing engines. The graph-based method offers flexibility and speed even on big maps. Both techniques were studied separately and provide good results. The missing part is how to combine these approaches to improve robustness of full \gls{SLAM} solution. To fulfill this goal, we will present a novel method for robust registration on top of \gls{NDT} grids. The most challenging part of this fusion is how to represent the map. We use method based on small local mini maps which are easily used in the graph of the \gls{SLAM} optimizer. Our algorithm has the additional robustness to odometry error by utilizing our \gls{NDT} version of incremental scan matching. The combination of these part creates the whole system which can estimate its position without initial guess and robustly close errors caused by imprecise robot movement. On top of algorithm benefits, we wanted to make source code and implementation easily accessible and improvable. For this reason, we have decided to implement it in \gls{ROS}, which is a current standard environment for robotic projects of all sizes.

This work has following structure. First chapter analyze full \gls{SLAM} on \gls{NDT} maps. The second chapter provides more information about algorithms used in this work. The third chapter describes the whole system of NDT SLAM. The fourth chapter makes the focus on implementation details behind the algorithms. In the last chapter, we wrap up results of this algorithm and compare it to existing \gls{ROS} implementations.  

